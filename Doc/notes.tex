\documentclass[11pt]{article}
\usepackage{natbib}
\begin{document}  
%     
%Here are some macro's saving time and labour:     
%     
\newcommand{\const}{\mbox{const}}      
\newcommand{\est}{\mbox{{\tiny est}}}      
\newcommand{\im}{\mbox{$\Im \mbox{m}$}}      
\newcommand{\obs}{\mbox{{\tiny obs}}}      
\newcommand{\otherwise}{\mbox{otherwise}}      
\newcommand{\real}{\mbox{$\Re \mbox{e}$}}      
\newcommand{\sign}{\mbox{sign}}      
\newcommand{\sinc}{\mbox{sinc}}      
%
\newcommand{\p}{\mbox{$\partial$}}      
\renewcommand{\d}{\mbox{$\partial$}}      
\newcommand{\w}{\mbox{$\omega$}}      
%
\newcommand{\AAA}{\mbox{\boldmath $A$}}   
\newcommand{\BB}{\mbox{\boldmath $B$}}     
\newcommand{\CC}{\mbox{\boldmath $C$}}     
\newcommand{\DD}{\mbox{\boldmath $D$}}     
\newcommand{\EE}{\mbox{\boldmath $E$}}     
\newcommand{\FF}{\mbox{\boldmath $F$}}   
\newcommand{\GG}{\mbox{\boldmath $G$}}   
\newcommand{\HH}{\mbox{\boldmath $H$}}   
\newcommand{\II}{\mbox{\boldmath $I$}}   
\newcommand{\JJ}{\mbox{\boldmath $J$}}   
\newcommand{\KK}{\mbox{\boldmath $K$}}   
\newcommand{\LL}{\mbox{\boldmath $L$}}   
\newcommand{\MM}{\mbox{\boldmath $M$}}   
\newcommand{\NN}{\mbox{\boldmath $N$}}   
\newcommand{\OO}{\mbox{\boldmath $O$}}   
\newcommand{\PP}{\mbox{\boldmath $P$}}   
\newcommand{\QQ}{\mbox{\boldmath $Q$}}   
\newcommand{\RR}{\mbox{\boldmath $R$}}   
\newcommand{\SSS}{\mbox{\boldmath $S$}}   
\newcommand{\TT}{\mbox{\boldmath $T$}}   
\newcommand{\UU}{\mbox{\boldmath $U$}}   
\newcommand{\VV}{\mbox{\boldmath $V$}}   
\newcommand{\WW}{\mbox{\boldmath $W$}}   
\newcommand{\XX}{\mbox{\boldmath $X$}}   
\newcommand{\YY}{\mbox{\boldmath $Y$}}   
\newcommand{\ZZ}{\mbox{\boldmath $Z$}}   
%
\newcommand{\aaa}{\mbox{\boldmath $a$}}     
\newcommand{\bb}{\mbox{\boldmath $b$}}     
\newcommand{\cc}{\mbox{\boldmath $c$}}     
\newcommand{\dd}{\mbox{\boldmath $d$}}     
\newcommand{\ee}{\mbox{\boldmath $e$}}   
\newcommand{\ff}{\mbox{\boldmath $f$}}   
%\newcommand{\ggg}{\mbox{\boldmath $g$}}   
\newcommand{\hh}{\mbox{\boldmath $h$}}   
\newcommand{\ii}{\mbox{\boldmath $i$}}   
\newcommand{\jj}{\mbox{\boldmath $j$}}   
\newcommand{\kk}{\mbox{\boldmath $k$}}   
%\newcommand{\lll}{\mbox{\boldmath $l$}}   
\newcommand{\mm}{\mbox{\boldmath $m$}}   
\newcommand{\nn}{\mbox{\boldmath $n$}}   
\newcommand{\pp}{\mbox{\boldmath $p$}}   
\newcommand{\qq}{\mbox{\boldmath $q$}}   
\newcommand{\rr}{\mbox{\boldmath $r$}}   
\newcommand{\sss}{\mbox{\boldmath $s$}}   
\newcommand{\ttt}{\mbox{\boldmath $t$}}   
\newcommand{\uu}{\mbox{\boldmath $u$}}   
\newcommand{\vv}{\mbox{\boldmath $v$}}   
\newcommand{\ww}{\mbox{\boldmath $w$}}   
\newcommand{\xx}{\mbox{\boldmath $x$}}   
\newcommand{\yy}{\mbox{\boldmath $y$}}   
\newcommand{\zz}{\mbox{\boldmath $z$}}   
%
\newcommand{\balpha}{\mbox{\boldmath $\alpha$}}     
\newcommand{\bpsi}{\mbox{\boldmath $\psi$}}     
\newcommand{\bphi}{\mbox{\boldmath $\phi$}}     
\newcommand{\bbeta}{\mbox{\boldmath $\beta$}}     
\newcommand{\btheta}{\mbox{\boldmath $\theta$}}     
\newcommand{\bdelta}{\mbox{\boldmath $\delta$}}     
\newcommand{\bgamma}{\mbox{\boldmath $d$}}     
\newcommand{\bGamma}{\mbox{\boldmath $\Gamma$}}     
\newcommand{\bLambda}{\mbox{\boldmath $\Lambda$}}     
\newcommand{\bmu}{\mbox{\boldmath $\mu$}}     
\newcommand{\bnabla}{\mbox{\boldmath $\nabla$}}     
\newcommand{\brho}{\mbox{\boldmath $\rho$}}     
\newcommand{\bSigma}{\mbox{\boldmath $\Sigma$}}     
\newcommand{\bsigma}{\mbox{\boldmath $\sigma$}}     
\newcommand{\bxi}{\mbox{\boldmath $\xi$}}     
\newcommand{\bepsilon}{\mbox{\boldmath $\epsilon$}}     
\newcommand{\blambda}{\mbox{\boldmath $\lambda$}}     
\newcommand{\BLambda}{\mbox{\boldmath $\Lambda$}}     
%-------------------------------------%
%  \Appendix - a new appendix command %
%-------------------------------------%
%The appendix command is used as in
% \Appendix{A}{The wave equation as a matrix equation}
\newcommand {\Appendix}[2]{
              \section*{APPENDIX #1: #2}
              \setcounter{equation}{0}
              \renewcommand{\theequation} 
              {A-\arabic{equation}}}
\newcommand {\Appendices}[2]{
              \section*{APPENDIX #1: #2 }
              \setcounter{equation}{0}
              \renewcommand{\theequation} 
              {#1-\arabic{equation}}}
%------------------------------------%
%    \aref - a new cite command.     % 
%------------------------------------%
\newcommand{\aref}[2]{\nocite{#1}#2} 
%----------------------------------------
%\eqref -an equation reference command
%----------------------------------------
\newcommand{\eqref}[1]{(\ref{#1})}
%\newcommand{\eqref}[1]{\ref{#1}}
  %bring in ... the macro's    
%
%---------------------------------------------------------------
\newcommand{\dt}[1]{\partial_t{#1}}
\newcommand{\ddt}[1]{\partial^2_t{#1}}
\newcommand{\dddt}[1]{\partial^3_t{#1}}
\newcommand{\inv}[1]{\frac{1}{#1}}
\newcommand{\diff}[1]{D^+_{#1}}
\newcommand{\difb}[1]{D^-_{#1}}
%---------------------------------------------------------------
\title{Visco-elastic Finite difference seismic modeling with 
       simple and effective boundary conditions}
\author{B.Arntsen}
\date{\today}
\maketitle
\clearpage
PRELIMINARY VERSION UNDER DEVELOPMENT. CONTAINS ERRORS.
%===============================================================================
\section*{Elastic equations of motion} 
%===============================================================================
Consider an elastic medium characterized by the density $\rho(\xx)$, 
and lam\'{e}
parameters $\lambda(\xx)$ and $\mu(\xx)$ at each spatial point $\xx$.
Now assume that $u(\xx)_i$ is the i'th component of the particle displacement 
and
$\sigma_{ij}$ is the stress tensor and $c_{ijkl}$ is Hook's tensor.
The elastic equations of motion and Hook's law in Cartesian coordinates are
given as
%
\begin{eqnarray}
  \rho(\xx) \ddt{u_i(\xx, t)} & = & \p_j \sigma_{ij}(\xx, t) +f_i(\xx,t),\\
					   \label{eq:el-eqm}
        \sigma_{ij}(\xx,t)    & = & c_{ijkl}e_{kl} +q_{ij}.
					  \label{eq:el-cse}
\end{eqnarray}
%
Here $f_i$ is a driving force, and $q_{ij}$ is a driving stress.
In the isotropic case one has
%
\begin{eqnarray}
   c_{ijkl} = \lambda\delta_{ij}\delta_{kl} + \mu\left(\delta_{ik}\delta_{jl} + \delta_{il}\delta_{jk}\right),
                                 \label{eq:el-moduli}
\end{eqnarray}
%
giving
%
\begin{eqnarray}
  \rho(\xx) \ddt{u_i(\xx, t)} & = & \p_j \sigma_{ij}(\xx, t) +f_i(\xx,t),\\
					   \label{eq:iso-elm}
  \sigma_{ij}(\xx, t) & = & 
        \lambda(\xx) e_{kk}\delta_{ij} + 2\mu e_{ij} +q_{ij}.
					  \label{eq:iso-cse}
\end{eqnarray}
%
Here $f_i$ is the i'th component of the driving force, while the strain
tensor $e_{ij}$
is equal to
%
\begin{eqnarray}
    e_{ij} = \frac{1}{2}[\p_i u_j(\xx, t) + \p_j u_i(\xx,t)].
					  \label{eq:iso-strain}
\end{eqnarray}
%
Writing out the individual components of the equations above, one gets:
%
\begin{eqnarray}
  \rho \ddt{u}_x = \p_x \sigma_{xx} +\p_y \sigma_{xy} +\p_z  \sigma_{xz} 
                                                      + f_x, \nonumber\\
  \rho \ddt{u}_y = \p_x \sigma_{yx} +\p_y \sigma_{yy} +\p_z  \sigma_{yz} 
                                                      + f_y, \nonumber\\
  \rho \ddt{u}_z = \p_x \sigma_{zx} +\p_y \sigma_{zy} +\p_z  \sigma_{zz} 
                                                      + f_z, \nonumber\\
\end{eqnarray}
%
\begin{eqnarray}
  \sigma_{xx} & = & \lambda \left (e_{xx} + e_{yy} + e_{zz}\right)+ 2\mu e_{xx} 
                                                   +q_{xx}, \nonumber\\
  \sigma_{yy} & = & \lambda \left (e_{xx} + e_{yy} + e_{zz}\right)+ 2\mu e_{yy} 
                                                   +q_{yy}, \nonumber\\
  \sigma_{zz} & = & \lambda \left (e_{xx} + e_{yy} + e_{zz}\right)+ 2\mu e_{zz} 
                                                   +q_{zz}, \nonumber\\
  \sigma_{xy} & = & 2\mu e_{xy} +q_{xy}, \nonumber\\
  \sigma_{xz} & = & 2\mu e_{xz} +q_{xz}, \nonumber\\
  \sigma_{yz} & = & 2\mu e_{yz} +q_{yz}. \nonumber\\ 
\end{eqnarray}
%
%===============================================================================
\section*{Viscoelastic equations of motion} 
%===============================================================================
Using the description in Appendix A, we can write down the viscoe-elastic equations
of motion and the constitutive relation.
In addition to visco-elastic stress-strain relation we have also
introduced time-dependence for the density.
The density then shows relaxation in the same way as a visco-elastic medium.
\begin{eqnarray}
  \ddt{u_i(\xx, t)} & = & \rho^{-1}_u(\xx)\p_j \sigma_{ij}(\xx, t) 
                                               +f_i(\xx,t),     \nonumber\\
                    & + & \chi(\xx,t)*\p_j \sigma_{ij}(\xx, t), \\
					   \label{eq:iso-ve-eqm}
  \sigma_{ij}(\xx, t) & = & 
                            \lambda_u e_{kk}\delta_{ij} + 2\mu_u e_{ij} 
                                               +q_{ij}          \nonumber\\
                      & + & \delta_{ij}\phi_{\lambda}(t)*e_{mm} 
                                         +2\phi_{\mu}(t)*e_{ij}
					  \label{eq:iso-ve-cse}
\end{eqnarray}
Writing out the individual components of the equations above, one gets:
%
\begin{eqnarray}
  \ddt{u}_x & = & \rho^{-1}_u\left[\p_x \sigma_{xx} +\p_y \sigma_{xy} 
                                 +\p_z \sigma_{xz}\right] + f_x,  \nonumber \\
            & = & + \chi*\left[\p_x \sigma_{xx} +\p_y \sigma_{xy} 
                                 +\p_z \sigma_{xz}\right],        \nonumber \\
  \ddt{u}_y & = & \rho^{-1}_u\left[\p_x \sigma_{yx} +\p_y \sigma_{yy} 
                                 +\p_z \sigma_{yz}\right] + f_y, \nonumber  \\
            & = & \chi*\left[\p_x \sigma_{yx} +\p_y \sigma_{yy} 
                                 +\p_z \sigma_{yz}\right] \nonumber         \\
  \ddt{u}_z & = & \rho^{-1}_u\left[\p_x \sigma_{zx} +\p_y \sigma_{zy} 
                                 +\p_z \sigma_{zz}\right] + f_z, \nonumber  \\
            & = & \chi*\left[\p_x \sigma_{zx} +\p_y \sigma_{zy} 
                                 +\p_z \sigma_{zz}\right].         \nonumber\\
\end{eqnarray}
%
\begin{eqnarray}
  \sigma_{xx} & = & \lambda_u \left (e_{xx} + e_{yy} + e_{zz}\right)
                                   + 2\mu_u e_{xx} +q_{xx}         \nonumber\\  
              & + & \phi_{\lambda}*[e_{xx}+e_{yy}+e_{zz}] 
                                    + 2\phi_{\mu}(t)*e_{xx},       \nonumber\\
  \sigma_{yy} & = & \lambda_u \left (e_{xx} + e_{yy} + e_{zz}\right)
                                    + 2\mu_u e_{yy} +q_{yy}        \nonumber\\ 
              & + & \phi_{\lambda}*[e_{xx}+e_{yy}+e_{zz}] 
                                    + 2\phi_{\mu}(t)*e_{yy},       \nonumber\\
  \sigma_{zz} & = & \lambda_u \left (e_{xx} + e_{yy} + e_{zz}\right)
                                    + 2\mu_u e_{zz} +q_{zz}        \nonumber\\ 
              & + & \phi_{\lambda}*[e_{xx}+e_{yy}+e_{zz}] 
                                    + 2\phi_{\mu}(t)*e_{zz},       \nonumber\\
  \sigma_{xy} & = & 2\mu_u e_{xy} +q_{xy} + 2\phi_{\mu}(t)*e_{xy}, \nonumber\\
  \sigma_{xz} & = & 2\mu_u e_{xz} +q_{xz} + 2\phi_{\mu}(t)*e_{xz}, \nonumber\\
  \sigma_{yz} & = & 2\mu_u e_{yz} +q_{yz}+  2\phi_{\mu}(t)*e_{yz}. \nonumber\\
\end{eqnarray}
%
%===============================================================================
\subsection*{Viscoelastic velocity-stress formulation} 
%===============================================================================
Using the velocity $v_i = \dot{u}_i$, one gets
\begin{eqnarray}
  \dt{v}_x & = & \rho^{-1}_u\left[\p_x \sigma_{xx} +\p_y \sigma_{xy} 
                                    +\p_z \sigma_{xz}\right] + f_x, \nonumber\\
            & = & + \chi*\left[\p_x \sigma_{xx} +\p_y \sigma_{xy} 
                                    +\p_z \sigma_{xz}\right],       \nonumber\\
  \dt{v}_y & = & \rho^{-1}_u\left[\p_x \sigma_{yx} +\p_y \sigma_{yy} 
                                    +\p_z \sigma_{yz}\right] + f_y, \nonumber\\
            & = & \chi*\left[\p_x \sigma_{yx} +\p_y \sigma_{yy} 
                                    +\p_z \sigma_{yz}\right]        \nonumber\\
  \dt{v}_z & = & \rho^{-1}_u\left[\p_x \sigma_{zx} +\p_y \sigma_{zy} 
                                    +\p_z \sigma_{zz}\right] + f_z, \nonumber\\
            & = & \chi*\left[\p_x \sigma_{zx} +\p_y \sigma_{zy} 
                                    +\p_z \sigma_{zz}\right].
\end{eqnarray}
\begin{eqnarray}
  \dot{\sigma}_{xx} 
     & = & \lambda_u \left (\dot{e}_{xx} 
                 + \dot{e}_{yy} + \dot{e}_{zz}\right)
                 + 2\mu_u \dot{e}_{xx} +\dot{q}_{xx},          \nonumber\\ 
     & + & \phi_{\lambda}*[\dot{e}_{xx}+\dot{e}_{yy}
                 +\dot{e}_{zz}] + 2\phi_{\mu}(t)*\dot{e}_{xx}, \nonumber\\
  \dot{\sigma}_{yy} 
     & = & \lambda_u \left (\dot{e}_{xx} 
                 +\dot{e}_{yy} + \dot{e}_{zz}\right)
                 + 2\mu_u \dot{e}_{yy} +\dot{q}_{yy}           \nonumber\\
     & + & \phi_{\lambda}*[\dot{e}_{xx}+\dot{e}_{yy}
                 +\dot{e}_{zz}] + 2\phi_{\mu}(t)*\dot{e}_{yy}, \nonumber\\
  \dot{\sigma}_{zz} 
    & = & \lambda_u \left (\dot{e}_{xx} 
                + \dot{e}_{yy} + \dot{e}_{zz}\right)
                + 2\mu_u \dot{e}_{zz} +q_{zz},                 \nonumber\\
    & + & \phi_{\lambda}*[\dot{e}_{xx}
                +\dot{e}_{yy}+\dot{e}_{zz}] 
                + 2\phi_{\mu}(t)*\dot{e}_{zz},                 \nonumber\\
  \dot{\sigma}_{xy} 
   & = & 2\mu_u \dot{e}_{xy} +q_{xy} 
                + 2\phi_{\mu}(t)*\dot{e}_{xy},                 \nonumber\\
  \dot{\sigma}_{xz} 
   & = & 2\mu_u \dot{e}_{xz} +q_{xz} 
                + 2\phi_{\mu}(t)*\dot{e}_{xz},                 \nonumber\\
  \dot{\sigma}_{yz} 
   & = & 2\mu_u \dot{e}_{yz} +q_{yz}
               +  2\phi_{\mu}(t)*\dot{e}_{yz}.                 \nonumber\\
\end{eqnarray}
%===============================================================================
\subsection*{Memory functions} 
%===============================================================================
We now define so-called memory variables by including the time convolution into one set of variables:
\begin{eqnarray}
 \gamma^l_{\lambda}(t) 
  & =& \frac{1}{\Delta\lambda}_l \phi^l_{\lambda}
       *[\dot{e}_{xx}+\dot{e}_{yy}+\dot{e}_{zz}]\label{eq:gamma-l} \\ 
 \gamma^l_{ij}(t) 
  & =& \frac{1}{\Delta\mu_l}\phi^l_{\mu}
       *[\dot{e}_{ij}],                         \label{eq:gamma-l} \\
 \theta^l_{kij}(t) 
  & =& \frac{1}{\Delta\rho^{-1}_l}\chi^l*\p_k[\sigma_{ij}],  
                                                \label{eq:theta-l}
\end{eqnarray}
Here 
\begin{eqnarray}
  \Delta\rho^{-1}_l 
     & = & \rho^{-1}_u\left(1-\frac{\tau^{\rho}_{\epsilon l}}
                                   {\tau^{\rho}_{\sigma l}}\right),    \\
 \Delta\lambda_l                              
     & = & \lambda_u\left(1-\frac{\tau^{\lambda}_{\epsilon l}}
                              {\tau^{\lambda}_{\sigma l}}\right),      \\
 \Delta\mu_l                              
     & = & \mu_u\left(1-\frac{\tau^{\mu}_{\epsilon l}}
                              {\tau^{\rho}_{\sigma l}}\right).         
\end{eqnarray}
This gives the expressions for the $\gamma$ functions as:
\begin{eqnarray}
 \gamma^l_{\lambda}    
    & = & \left[\frac{\exp(-t/\tau^{\lambda}_{\sigma l})}
                     {\tau^{\lambda}_{\sigma l}
                     \sum_{l=1}^N \frac{\tau^{\lambda}_{\epsilon l}}
                     {\tau^{\lambda}_{\sigma l}}}\right] 
                     [\dot{e}_{xx}+\dot{e}_{yy}+\dot{e}_{zz}],       \\
 \gamma^l_{ij}    
   & = & \left[\frac{\exp(-t/\tau^{\mu}_{\sigma l})}
                     {\tau^{\mu}_{\sigma l}
                     \sum_{l=1}^N \frac{\tau^{\mu}_{\epsilon l}}
                     {\tau^{\mu}_{\sigma l}}}\right] 
                     \dot{e}_{ij},                                   \\
 \theta^l_{kij}    
   & = & \left[\frac{\exp(-t/\tau^{\rho}_{\sigma l})}
                    {\tau^{\rho}_{\sigma l}
                    \sum_{l=1}^N \frac{\tau^{\rho}_{\epsilon l}}
                    {\tau^{\rho}_{\sigma l}}}\right] 
                    \p_k \sigma_{ij}.
\end{eqnarray}

and also 
\begin{eqnarray}
  \gamma_{\lambda}(t) & = & \sum_{l=1}^N \gamma^l_{\lambda},   \nonumber\\
  \gamma_{ij}(t)      & = & \sum_{l=1}^N \gamma^l_{ij},        \nonumber\\
  \theta_{kij}(t)     & = & \sum_{l=1}^N \theta^l_{ij}.        \nonumber
                           \label{eq:gamma}
\end{eqnarray}

This gives the final form of the viscoelastic equations
%
\begin{eqnarray}
  \dt{v}_x 
     & = & \rho^{-1}_i\left(\p_x \sigma_{xx} +\p_y 
           \sigma_{xy} +\p_z \sigma_{xz}\right) + f_x,       \nonumber\\
     & + & \sum_{l=0}^N\theta^l_{xxx}\Delta\rho^{-1}_{l} 
           +  \sum_{l=0}^N\theta^l_{yxy}\Delta\rho^{-1}_{l} 
           +  \sum_{l=0}^N\theta^l_{zxz}\Delta\rho^{-1}_{l}, \nonumber\\ 
  \dt{v}_y 
     & = & \p_x \sigma_{yx} +\p_y \sigma_{yy} +\p_z 
           \sigma_{yz} + f_y,                                \nonumber\\
     & + & \sum_{l=0}^N\theta^l_{xyx}\Delta\rho^{-1}_{l} 
           +  \sum_{l=0}^N\theta^l_{yyy}\Delta\rho^{-1}_{ll} 
           +  \sum_{l=0}^N\theta^l_{zyz}\Delta\rho^{-1}_{ll}.\nonumber\\ 
  \dt{v}_z 
    & = & \p_x \sigma_{zx} +\p_y \sigma_{zy} 
          +\p_z \sigma_{zz} + f_z,                           \nonumber\\
    & + & \sum_{l=0}^N\theta^l_{xzx} +  \sum_{l=0}^N\theta^l_{yzy} +  
          \sum_{l=0}^N\theta^l_{zzz}.                        \nonumber\\
\end{eqnarray}
\begin{eqnarray}
  \dot{\sigma}_{xx} 
    & = & \lambda_u \left (\dot{e}_{xx} + \dot{e}_{yy} + \dot{e}_{zz}\right)
         + 2\mu_u \dot{e}_{xx} +\dot{q}_{xx}                 \nonumber\\
    & + & \sum_{l=1}^N\gamma^l_{\lambda}\Delta\lambda_l  
         + 2\sum_{l=1}^N\gamma^l_{xx}\Delta\mu_l,            \nonumber\\
  \dot{\sigma}_{yy} 
    & = & \lambda_u 
         \left (\dot{e}_{xx} + \dot{e}_{yy} + \dot{e}_{zz}\right)
         + 2\mu_u \dot{e}_{yy} +\dot{q}_{yy},                     \nonumber\\
    & + & \sum_{l=1}^N\gamma^l_{\lambda}\Delta\lambda_l  
         + 2\sum_{l=1}^N\gamma^l_{yy}\Delta\mu_l,                 \nonumber\\
  \dot{\sigma}_{zz} 
    & = & \lambda_u \left (\dot{e}_{xx} + \dot{e}_{yy} 
         + \dot{e}_{zz}\right)+ 2\mu_u \dot{e}_{zz} +\dot{q}_{zz} \nonumber\\
    & + & \sum_{l=1}^N\gamma^l_{\lambda}\Delta\lambda_l  
         + 2\sum_{l=1}^N\gamma_{zz}\Delta\mu_l,                   \nonumber\\
  \dot{\sigma}_{xy} 
    & = & 2\mu \dot{e}_{xy} + 2\sum_{l=1}^N\gamma^l_{xy}
          \Delta\mu_l+\dot{q}_{xy}                                \nonumber,\\
  \dot{\sigma}_{xz} 
    & = & 2\mu \dot{e}_{xz} + 2\sum_{l=1}^N\gamma^l_{xz}
          \Delta\mu_l+\dot{q}_{xz}                                \nonumber,\\
  \dot{\sigma}_{yz} 
    & = & 2\mu \dot{e}_{yz} + 2\sum_{l=1}^N\gamma^l_{yz}
          \Delta\mu_l+\dot{q}_{yz}                                \nonumber,\\
\end{eqnarray} 
\begin{eqnarray}
    \dot{e_{xx}} & = & \p_x v_x, \nonumber\\
    \dot{e_{yy}} & = & \p_y v_y, \nonumber\\
    \dot{e_{zz}} & = & \p_z v_z, \nonumber\\
    \dot{e_{xy}} & = & \frac{1}{2}(\p_x v_y + \p_y v_x),\nonumber\\
    \dot{e_{xz}} & = & \frac{1}{2}(\p_x v_z + \p_z v_y),\nonumber\\
    \dot{e_{yz}} & = & \frac{1}{2}(\p_y v_z + \p_z v_y).\nonumber\\
\end{eqnarray}

%===============================================================================
\section*{Integration of memory functions}
%===============================================================================
The memory functions  obeys approximately the relations

\begin{eqnarray}
\gamma^l_{\lambda}(t) & = &
            \exp(-\frac{\Delta t}{\tau^{\lambda l}_{\sigma}})\gamma^l_{\lambda}(t-\Delta t)
            +\left(\frac{\Delta t}{\tau^{\lambda}_{\sigma l}
            \sum_{l=1}^N\frac{\tau^{\lambda}_{\epsilon l}}{\tau^{\lambda}_{\sigma l}}}\right)
             \left(\dot{e}_{xx}+ \dot{e}_{yy}+\dot{e}_{zz}\right) ,\nonumber\\
\gamma^l_{ij}(t) & = &
            \exp(-\frac{\Delta t}{N\tau^{\mu}_{\sigma l}})\gamma^l_{ij}(t-\Delta t)
            + \left(\frac{\Delta t}{\tau^{\mu}_{\sigma l}
            \sum_{l=1}^N\frac{\tau^{\mu}_{\epsilon l}}{\tau^{\mu}_{\sigma l}}}\right) \dot{e}_{ij}.\nonumber\\
\theta^l_{kij}(t) & = &
            \exp(-\frac{\Delta t}{N\tau^{\rho}_{\sigma l}})\theta^l_{kij}(t-\Delta t)
            + \left(\frac{\Delta t}{\tau^{\rho}_{\sigma l}
            \sum_{l=1}^N\frac{\tau^{\rho}_{\epsilon l}}{\tau^{\mu}_{\sigma l}}}\right) \p_k \sigma_{ij}.\nonumber\\
\end{eqnarray}
%
Defining the quantities
\begin{eqnarray}
\alpha^l_1 & = & \exp(-\frac{\Delta t}{\tau^{\lambda}_{\sigma l}}),\\
\alpha^l_2 & = & \frac{\Delta t}{\tau^{\lambda}_{\sigma l}\sum_{l=1}^N\frac{\tau^{\lambda}_{\epsilon l}}{\tau^{\lambda}_{\sigma l}}} ,\\
\beta^l_1  & = &\exp(-\frac{\Delta t}{\tau^{\mu}_{\sigma l}}) ,\\
\beta^l_2  & = &\frac{\Delta t}{\tau^{\mu}_{\sigma l}\sum_{l=1}^N\frac{\tau^{\mu}_{\epsilon l}}{\tau^{\mu}_{\sigma l}}},\\
\eta^l_1  & = &\exp(-\frac{\Delta t}{\tau^{\rho}_{\sigma l}}) ,\\
\eta^l_2  & = &\frac{\Delta t}{\tau^{\rho}_{\sigma l}\sum_{l=1}^N\frac{\tau^{\rho}_{\epsilon l}}{\tau^{\rho}_{\sigma l}}} 
\end{eqnarray}

%
we get
\begin{eqnarray}
\gamma^l_{\lambda}(t)  & = &\alpha^l_1 \gamma^l_{\lambda}(t-\Delta t) 
                        + \alpha^l_2\left( \dot{e}_{xx}+ \dot{e}_yy+\dot{e}_zz\right) ,\nonumber \\
\gamma^l_{ij}(t)      & = &\beta^l_1 \gamma^l_{ij}(t - \Delta t) 
                        + \beta^l_2\dot{e}_{ij}.\nonumber\\
\theta^l_{kij}(t)      & = &\eta^l_{1} \theta^l_{kij}(t - \Delta t) 
                        + \eta^l_2\p_k \sigma_{ij}.\nonumber\\
\end{eqnarray}
%
%================================================================================
\section*{Discretization of the three dimensional case}
%================================================================================
We now consider a regular grid with positions $\xx$ defined as
\begin{eqnarray}
\xx & = & (x,y,z),\\
x   & = & p\Delta x,\\
y   & = & q\Delta y,\\
z   & = & r\Delta z,\\
t   & = & n\Delta t.
\end{eqnarray}
where $p=0,1,2,\ldots,N_x$, $q=0,1,2,\ldots,N_y$ and $r=0,1,2,\ldots,N_z$ and
$N_x,N_y$ and $N_z$ are the number of grid points in the $x,y$ and $z$-directions.
$n=0,1,2,\ldots,N_t$ where $N_t$ is the number of time steps.

We will also need a regular grid which is displaced, or staggered, relative to the the regular grid.
Sometimes we will refer to the regular grid as the reference grid.

The particle velocities  $v_x$,$v_y$ and $v_z$ are defined on staggered grids
as follows
%
\begin{eqnarray}
   v_x(\xx, t) & = & v_x(x+\Delta x/2, y, z, t),\\
   v_y(\xx, t) & = & v_y(x, y+\Delta y/2, z, t),\\
   v_z(\xx, t) & = & v_z(x, y, z+\Delta z/2, t). 
\end{eqnarray}
%
The $\theta_{kij}$ are defined on the same staggered grid as the
particle velocities:
\begin{eqnarray}
   \theta_{xxx}(\xx, t) & = & \theta(x+\Delta x/2,y,z)_{xxx},\\
   \theta_{yxy}(\xx, t) & = & \theta(x,y+\Delta y/2,z)_{yxy},\\
   \theta_{zxz}(\xx, t) & = & \theta(x,y,z+\Delta z/2)_{zxz},\\
   \theta_{xyx}(\xx, t) & = & \theta(x+\Delta x/2,y,z)_{xyx},\\
   \theta_{yyy}(\xx, t) & = & \theta(x,y+\Delta y/2,z)_{yyy},\\
   \theta_{zyz}(\xx, t) & = & \theta(x,y,z+\Delta z/2  )_{zyz},\\
   \theta_{xzx}(\xx, t) & = & \theta(x+\Delta x/2,y,z)_{xzx},\\
   \theta_{yzy}(\xx, t) & = & \theta(x,y+\Delta y/2,z)_{yzy}\\
   \theta_{zzz}(\xx, t) & = & \theta(x,y,z+\Delta z/2)_{xxx}.
\end{eqnarray}


The diagonal stresses and the gamma functions are defined on a regular grid:
\begin{eqnarray}
   \sigma_{xx}(\xx, t) & = & \sigma_{xx}( x, y,z, t), \\
   \sigma_{yy}(\xx, t) & = & \sigma_{yy}(x, y, z, t), \\
   \sigma_{zz}(\xx, t) & = & \sigma_{zz}(x, y, z, t), \\ 
   \gamma_{\lambda}(\xx, t) & = & \gamma_{\lambda}(x, y,  z, t), \\ 
   \gamma_{\mu}(\xx, t) & = & \gamma_{\mu}(x, y, z, t), \\ 
   \gamma_{xx}(\xx, t) & = & \gamma_{xx}(x, y, z,  t), \\ 
   \gamma_{yy}(\xx, t) & = & \gamma_{yy}( x, y, z,  t), \\ 
   \gamma_{zz}(\xx, t) & = & \gamma_{zz}(x, y, z,  t). 
\end{eqnarray}
%
The off-diagonal stresses and the off-diagonal gamma functions are defined on
several staggered grids
%
\begin{eqnarray}
   \sigma_{xy}(\xx, t) & = & \sigma_{xy}(x +\Delta x/2, y + \Delta y/2, z, t), \\
   \sigma_{xz}(\xx, t) & = & \sigma_{xz}(x +\Delta x/2, y,z + \Delta z/2,  t), \\
   \sigma_{yz}(\xx, t) & = & \sigma_{yz}(x, y + \Delta y/2, z+\Delta z/2,  t), \\
   \gamma_{xy}(\xx, t) & = & \gamma_{xy}(x +\Delta x/2, y + \Delta y/2, z, t), \\
   \gamma_{xz}(\xx, t) & = & \gamma_{xz}(x +\Delta x/2, y,z + \Delta z/2,  t), \\
   \gamma_{yz}(\xx, t) & = & \gamma_{yz}(x, y + \Delta y/2, z+\Delta z/2,  t), \\
\end{eqnarray}
%

The Lam\'{e} parameters are defined on a regular grid as follows:
as follows:
%
\begin{eqnarray}
   \lambda_u(\xx) & = & \lambda_u(x,y,z),\\
   \mu_u(\xx)     & = & \mu_u(x,y, z).
\end{eqnarray}

The inverse density and $\mu$ are defined  on three different staggered grids as follows:
\begin{eqnarray}
   \rho_x^{-1}(\xx) & = & \rho^{-1}(x+\Delta x/2, y, z),\\
   \rho_y^{-1}(\xx) & = & \rho^{-1}(x,y+\Delta y/2, z),\\
   \rho_z^{-1}(\xx) & = &\rho^{-1}(x,y,z+\Delta z/2),\\
   \mu_{uxy}(\xx) & = &   \mu_u(x+\Delta x/2, y+\Delta y/2, z),\\
   \mu_{uyz}(\xx) & = &   \mu_u(x, y+\Delta y/2, z+\Delta z/2),\\
   \mu_{uxz}(\xx) & = &   \mu_u(x+\Delta x/2, y,z+\Delta z),\\
\end{eqnarray}

The visco-elastic parameters $\beta_1$, $\beta_2$ and $\eta_1$ and $\eta_2$ are also defined on several staggered grids:
\begin{eqnarray}
   \beta_{1xy}\xx) & = &   \beta_1(x+\Delta x, y+\Delta y, z),\\
   \beta_{1yz}(\xx) & = &  \beta_1(x,y+\Delta y, z+\Delta z),\\
   \beta_{1xz}(\xx) & = &  \beta_1(x+\Delta x, y,z+\Delta z),\\
   \beta_{2xy}(\xx) & = &  \beta_2(x+\Delta x, y+\Delta y, z),\\
   \beta_{2yz}(\xx) & = &  \beta_2(x,y+\Delta y, z+\Delta z),\\
   \beta_{2xz}(\xx) & = &  \beta_2(x+\Delta x, y,z+\Delta z),\\
   \eta_{1x}(\xx) & = &   \eta_1(x+\Delta x/2,y,z),\\
   \eta_{2x}(\xx) & = &   \eta_2(x+\Delta x/2,y,z),\\
   \eta_{1y}(\xx) & = &   \eta_1(x,y+\Delta y/2,z),\\
   \eta_{2y}(\xx) & = &   \eta_1(x,y+\Delta y/2,z),\\
   \eta_{1z}(\xx) & = &   \eta_1(x,y,z+\Delta z/2),\\
   \eta_{2z}(\xx) & = &   \eta_1(x,y,z+\Delta z/2).
\end{eqnarray}

Differentiation is now replaced by numerical approximations so that $\p_x,\p_y$ and $\p_z$ are
replaced with numerical operators $d^+_x,d^-_x,d^+_y,d^-_y,d^+_z$ and $d^-_z$.
These operators connects the staggered and reference grids, and we illustrate this with the differentiation
in the x-direction.
The derivative of a function $a(x)$ is approximately given at $a(x+\Delta x/2)$ and
at $a(x-\Delta x/2)$ by
\begin{eqnarray}
a'(x+\Delta x/2) & = & d^+_x a(x),\\
a'(x-\Delta x/2) & = & d^-_x a(x).
\end{eqnarray}

The differentiators $d^+$ and $d^-$ are given by
\citep{Holberg1987}
%
\begin{eqnarray}
   \d^+ & = & \inv{\Delta x}\sum_{l=1}^L \alpha_l\left[u(x +l\Delta x) -
		          u(x - (l-1)\Delta x\right]\nonumber\\	
   \d^- & = & \inv{\Delta x}\sum_{l=1}^L \alpha_l\left[u(x +(l-1)\Delta x) -
					          u(x - l\Delta x)\right]
\end{eqnarray}
where the coefficients $\alpha_l$ are found through an optimization procedure.
Similar differentiators are defined for the $y$-direction and for the $z$-direction, with obvious names.

Using the numerical differentiators the equations of motion becomes:

\begin{eqnarray}
  \dt{v}_x & = & \rho^{-1}_i\left(d^+_x \sigma_{xx} +d^+_y \sigma_{xy} +d^+_z \sigma_{xz}\right) + f_x \\
           & + & \sum_{l=0}^N\theta^l_{xxx} +  \sum_{l=0}^N\theta^l_{yxy} +  \sum_{l=0}^N\theta^l_{zxz}, \\ 
  \dt{v}_y & = & d^+_x \sigma_{yx} +d^+_y \sigma_{yy} +d^+_z \sigma_{yz} + f_y, \\
           & + & \sum_{l=0}^N\theta^l_{xyx} +  \sum_{l=0}^N\theta^l_{yyy} +  \sum_{l=0}^N\theta^l_{zyz}, \\ 
  \dt{v}_z & = & d^+_x \sigma_{zx} +d^+_y \sigma_{zy} +d^+_z \sigma_{zz} + f_z, \\
           & + & \sum_{l=0}^N\theta^l_{xzx} +  \sum_{l=0}^N\theta^l_{yzy} +  \sum_{l=0}^N\theta^l_{zzz}.
\end{eqnarray}

The computation of the strains becomes as follows:
\begin{eqnarray}
    \dot{e}_{xx} & = & d^-_x v_x, \nonumber \\
    \dot{e}_{yy} & = & d^-_y v_y, \nonumber \\
    \dot{e}_{zz} & = & d^-_z v_z, \nonumber \\
    \dot{e}_{xy} & = & \frac{1}{2}(d^+_x v_y + d^+_y v_x),\nonumber\\
    \dot{e}_{xz} & = & \frac{1}{2}(d^+_x v_z + d^+_z v_x),\nonumber\\
    \dot{e}_{yz} & = & \frac{1}{2}(d^+_y v_z +d^+_z v_y).\nonumber\\
                \label{eq:strains}
\end{eqnarray}
The time derivatives is approximated by the central difference
\begin{eqnarray}
\dot{a}(t) = \frac{a(t+\Delta t/2) - a(t-\Delta t/2)}{\Delta t}
                      \label{eq:time-derivative}
\end{eqnarray}

%=================================================================
\section*{Solution algorithm for the three dimensional case}
%=================================================================
We are now in a position to formulate a complete numerical 
solution of the visco-elastic equations.

\subsection*{Computation of the particle velocity}
We use the expression for the approximate time derivative given by equation \eqref{eq:time-derivative}
in equations \eqref{eq:vel-discrete} to obtain an expression for the components of the particle velocity
\begin{eqnarray}
v_x(t+\Delta t/2) & = & \Delta t\rho_x^{-1}\left[d^+_x \sigma_{xx}(t) +d^+_y \sigma_{xy}(t) +d^+_z \sigma_{xz}(t)\right] 
                         + \Delta t \rho_x^{-1}f_x(t) +\nonumber\\ 
           & + & \Delta t \sum_{l=0}^N\theta^l_{xxx}(t) +  \Delta t \sum_{l=0}^N\theta^l_{yxy}(t) +  \Delta t \sum_{l=0}^N\theta^l_{zxz}(t)
                   +v_x(t-\Delta t/2),\nonumber\\
v_y(t+\Delta t/2) & = & \Delta t\rho_y^{-1}\left[d^+_x \sigma_{yx}(t) +d^+_y \sigma_{yy}(t) +d^+_z \sigma_{yz}(t)\right] 
                         + \Delta t \rho_y^{-1}f_y(t)\nonumber\\
           & + & \Delta t\sum_{l=0}^N\theta^l_{xyx}(t) +  \Delta t\sum_{l=0}^N\theta^l_{yyy}(t) +  \Delta t\sum_{l=0}^N\theta^l_{zyz}(t)
                        + v_y(t-\Delta t/2), \nonumber\\ 
v_z(t+\Delta t/2) & = & \Delta t \rho_z^{-1}\left[d^+_x \sigma_{zx}(t) +d^+_y \sigma_{zy}(t) +d^+_z \sigma_{zz}(t)\right] 
                        + \Delta t \rho_z^{-1}f_z(t)\nonumber\\  
           & + & \Delta t\sum_{l=0}^N\theta^l_{xzx}(t) +  \Delta t\sum_{l=0}^N\theta^l_{yzy}(t) +  \Delta t\sum_{l=0}^N\theta^l_{zzz}(t)\nonumber.
                       +v_z(t-\Delta t/2) \\  
\end{eqnarray}
The strains can now be computed from equation \eqref{eq:strains}
\begin{eqnarray}
    \dot{e}_{xx}(t+\Delta t/2) & = & d^-_x v_x(t+\Delta t/2), \nonumber\\
    \dot{e}_{yy}(t+\Delta t/2) & = & d^-_y v_y(t+\Delta t/2), \nonumber\\
    \dot{e}_{zz}(t+\Delta t/2) & = & d^-_z v_z(t+\Delta t/2), \nonumber\\
    \dot{e}_{xy}(t+\Delta t/2) & = & \frac{1}{2}\left[d^+_x v_y(t+\Delta t/2) + d^+_y v_x(t+\Delta t/2)\right],\nonumber \\
    \dot{e}_{xz}(t+\Delta t/2) & = & \frac{1}{2}\left[d^+_x v_z(t+\Delta t/2) + d^+_z v_x(t+\Delta t/2)\right],\nonumber \\
    \dot{e}_{yz}(t+\Delta t/2) & = & \frac{1}{2}\left[d^+_y v_z(t+\Delta t/2) +d^+_z v_y(t+\Delta t/2)\right]. \nonumber\\
\end{eqnarray}
%
Equations \eqref{eq:stress}   can be solved for the stresses using the same approach as for the particle velocities:
\begin{eqnarray}
  \sigma_{xx}(t+\Delta t) & = & \Delta t\lambda_u \left [\dot{e}_{xx}(t+\Delta t/2) + \dot{e}_{yy}(t+\Delta t/2) + 
                                \dot{e}_{zz}(t+\Delta t/2)\right]\nonumber\\
                           & + & 2\mu_u \dot{e}_{xx}(t+\Delta t/2) +\Delta t\dot{q}_{xx} \nonumber\\
                           & + & \Delta t\sum_{l=1}^N\gamma^l_{\lambda}(t+\Delta t/2)\Delta\lambda_l  
                             +   2\Delta t\sum_{l=1}^N\gamma^l_{xx}(t+\Delta t/2)\Delta\mu_l(t+\Delta t/2)\nonumber\\ 
                           & + &\sigma_{xx}(t),\\
  \sigma_{yy}(t+\Delta t) & = & \Delta t\lambda_u \left [\dot{e}_{xx}(t+\Delta t/2) + \dot{e}_{yy}(t+\Delta t/2) + 
                                \dot{e}_{zz}(t+\Delta t/2)\right]\\
                          & + & 2\Delta t \mu_u \dot{e}_{yy}(t+\Delta t/2) +\Delta t \dot{q}_{yy}(t+\Delta t/2) \nonumber\\
                          & + & \Delta t\sum_{l=1}^N\gamma^l_{\lambda}(t+\Delta t/2)\Delta\lambda_l  
                            +   2\Delta t\sum_{l=1}^N\gamma^l_{yy}(t+\Delta t/2)\Delta\mu_l\\
                          & +& \sigma_{yy}(t) \\
  \sigma_{zz}(t+\Delta t) & = & \Delta t\lambda_u \left [\dot{e}_{xx}(t+\Delta t/2) + \dot{e}_{yy}(t+\Delta t/2) 
                               + \dot{e}_{zz}(t+\Delta t/2)\right] \\
                          & + & 2\mu_u \dot{e}_{zz}(t+\Delta t/2) +\Delta t \dot{q}_{zz}(t+\Delta t/2) \nonumber\\
                          & + & \Delta t\sum_{l=1}^N\gamma^l_{\lambda}(t+\Delta t/2)\Delta\lambda_l   
                            +   2\Delta t\sum_{l=1}^N\gamma_{zz}\Delta\mu_l (t+\Delta t/2)\nonumber, \\
                          & + & \sigma_{zz}.\\
  \sigma_{xy}(t+\Delta t) & = & 2\Delta t\mu \dot{e}_{xy}(t+\Delta t/2) + 2\Delta t\sum_{l=1}^N\gamma^l_{xy}(t+\Delta t/2)\Delta\mu_l+
                                \Delta t \dot{q}_{xy}(t+\Delta t/2)\nonumber\\
                          & + & \sigma_{xy}(t)\nonumber,\\
  \sigma_{xz}(t+\Delta t) & = & 2\Delta t\mu \dot{e}_{xz}(t+\Delta t/2) + 2\Delta t \sum_{l=1}^N\gamma^l_{xz}(t+\Delta t/2)\Delta\mu_l+
                                    \Delta t \dot{q}_{xz}(t+\Delta t/2)\nonumber\\ 
                          & +& \sigma_{xz}(t),\\
  \sigma_{yz}(t+\Delta t) & = & 2\Delta \mu \dot{e}_{yz}(t+\Delta t/2) + 2\Delta t\sum_{l=1}^N\gamma^l_{yz}(t+\Delta t/2)\Delta\mu_l
                                   +\dot{q}_{yz}(t+\Delta t/2)\nonumber\\
                          & +& \sigma_{yz}(t)\nonumber.\\
                          \label{eq:stress}
\end{eqnarray} 
The gamma functions are updated as follows
\begin{eqnarray}
\gamma^l_{\lambda}(t+3/2\Delta t)  & = &\alpha^l_1 \gamma^l_{\lambda}(t+\Delta t/2) 
                        + \alpha^l_2\left[ \dot{e}_{xx}(t+\Delta t/)+ \dot{e}_{yy}(t+\Delta t/2)
                        +\dot{e}_{zz}(t+\Delta t/2)\right] ,\nonumber \\
\gamma^l_{\mu}(t+3/2\Delta t)     & = &\beta^l_1 \gamma^l_{\mu}(t+\Delta t/2) 
                        + \beta^l_2\left[ \dot{e}_{xx}(t+\Delta t/)+ \dot{e}_{yy}(t+\Delta t/2)
                        +\dot{e}_{zz}(t+\Delta t/2)\right] ,\nonumber \\
\gamma^l_{xy}(t+3/2\Delta t)      & = &\beta^l_{1xy} \gamma^l_{xy}(t + \Delta t/2) 
                        + \beta^l_{2xy}\dot{e}_{xy}(t+\Delta t/2).\nonumber\\
\gamma^l_{xz}(t+3/2\Delta t)      & = &\beta^l_{1xz} \gamma^l_{xz}(t + \Delta t/2) 
                        + \beta^l_{2xz}\dot{e}_{xz}(t+\Delta t/2).\nonumber\\
\gamma^l_{yz}(t+3/2\Delta t)      & = &\beta^l_{1xz} \gamma^l_{yz}(t + \Delta t/2) 
                        + \beta^l_{2xz}\dot{e}_{yz}(t+\Delta t/2).\nonumber\\
\end{eqnarray}

The $\theta$ functions are updated as:
\begin{eqnarray}
   \theta_{xxx}(t+\Delta t)  
                         & = &\eta^l_{1} \theta^l_{xxx}(t + \Delta t/2)
                        + \eta^l_{2}\p_x{\sigma}_{xx}(t+\Delta t/2).\nonumber\\
   \theta_{yxy}(t+\Delta t)  
                         & = &\eta^l_{1} \theta^l_{yxy}(t + \Delta t/2)
                        + \eta^l_{2}\p_y{\sigma}_{xy}(t+\Delta t/2).\nonumber\\
   \theta_{zxz}(t+\Delta t)  
                         & = &\eta^l_{1} \theta^l_{zxz}(t + \Delta t/2)
                        + \eta^l_{2}\p_z {\sigma}_{xz}(t+\Delta t/2).\nonumber\\
   \theta_{xyx}(t+\Delta t)  
                         & = &\eta^l_{1} \theta^l_{xyx}(t + \Delta t/2)
                        + \eta^l_{2}\p_x {\sigma}_{yx}(t+\Delta t/2).\nonumber\\
   \theta_{yyy}(t+\Delta t)  
                         & = &\eta^l_{1} \theta^l_{yyy}(t + \Delta t/2)
                        + \eta^l_{2}\p_y {\sigma}_{yy}(t+\Delta t/2).\nonumber\\
   \theta_{zyz}(t+\Delta t)  
                         & = &\eta^l_{1} \theta^l_{zyz}(t + \Delta t/2)
                        + \eta^l_{2}\p_z {\sigma}_{yz}(t+\Delta t/2).\nonumber\\
   \theta_{xyx}(t+\Delta t)  
                         & = &\eta^l_{1} \theta^l_{xyx}(t + \Delta t/2)
                        + \eta^l_{2}\p_x {\sigma}_{yx}(t+\Delta t/2).\nonumber\\
   \theta_{xzx}(t+\Delta t)  
                         & = &\eta^l_{1} \theta^l_{xzx}(t + \Delta t/2)
                        + \eta^l_{2}\p_x {\sigma}_{zx}(t+\Delta t/2).\nonumber\\
   \theta_{yzy}(t+\Delta t)  
                         & = &\eta^l_{1} \theta^l_{yzy}(t + \Delta t/2)
                        + \eta^l_{2}\p_y {\sigma}_{zy}(t+\Delta t/2).\nonumber\\
   \theta_{zzz}(t+\Delta t)  
                         & = &\eta^l_{1} \theta^l_{zzz}(t + \Delta t/2)
                        + \eta^l_{2}\p_z {\sigma}_{zz}(t+\Delta t/2).\nonumber\\
\end{eqnarray}
%==============================================================================
\section*{2D Acoustic case}
%==============================================================================
For the acoustic 2D case we reduce the equations above by neglecting 
the y-axis terms
and putting $\mu=0$.
We consider also the pseudo-stress $\sigma$ defined by
\begin{eqnarray*}
\sigma = \frac{1}{2}\left(\sigma_{xx} + \sigma_{zz}\right) 
\end{eqnarray*}
We then get the acoustic 2D scheme as:
\begin{eqnarray*}
v_x(t+\Delta t/2) & = & \Delta t\left[\rho_{ux}^{-1}d^+_x \sigma_{xx}(t)  
                         + \rho_{ux}^{-1}f_x(t)\right] +\nonumber\\
           & + & \Delta t \sum_{l=0}^N\theta^l_{x}(t)\Delta\rho^{-1}_{x}  
                   +v_x(t-\Delta t/2),\nonumber\\
v_z(t+\Delta t/2) & = & \Delta t \left[\rho_{uz}^{-1}d^+_z \sigma_{zz}(t) 
                        + \Delta t \rho_{uz}^{-1}f_z(t)\right]\nonumber\\  
           & + & \Delta t\sum_{l=0}^N\theta^l_{z}(t)\Delta \rho^{-1}_{z}
                       +v_z(t-\Delta t/2). 
\end{eqnarray*}
The strains can now be computed from :
\begin{eqnarray*}
    \dot{e}_{xx}(t+\Delta t/2) & = & d^-_x v_x(t+\Delta t/2), \\
    \dot{e}_{zz}(t+\Delta t/2) & = & d^-_z v_z(t+\Delta t/2). 
\end{eqnarray*}
%
Equations \eqref{eq:stress}   can be solved for the stresses using 
the same approach as for the particle velocities:
\begin{eqnarray*}
  \sigma(t+\Delta t) 
       & = & \Delta t\lambda_u \left [\dot{e}_{xx}(t+\Delta t/2) 
             +\dot{e}_{zz}(t+\Delta t/2)\right]\nonumber
             +\Delta t\dot{q} \nonumber,                       \\
       & + & \Delta t\sum_{l=1}^N\gamma^l(t+\Delta t/2)\Delta\lambda_l  
                           + \sigma(t).
\end{eqnarray*} 

We now split the $\gamma^l$ into two parts $\gamma^l_x$ and $\gamma^l_z$ 
as follows:
\begin{eqnarray*}
  \sigma(t+\Delta t) & = & \Delta t\lambda_u \left [\dot{e}_{xx}(t+\Delta t/2) 
                          +\dot{e}_{zz}(t+\Delta t/2)\right]\nonumber
                           +\Delta t\dot{q} \nonumber\\
                     & + & \Delta t
                 \sum_{l=1}^N\left[\gamma^l_{x}(t+\Delta t/2)\Delta\lambda_l  
                                                   +\gamma^l_{z}(t+\Delta t/2)\Delta\lambda_l  \right]
                           + \sigma(t).\\
\end{eqnarray*} 

The $\theta$ functions are updated as:
\begin{eqnarray*}
   \theta_{x}(t+\Delta t)  
                         & = &\eta^l_{1x} \theta^l_{x}(t)
                        + \eta^l_{2x}\p_x{\sigma}(t),\\
   \theta_{z}(t+\Delta t)  
                         & = &\eta^l_{1z} \theta^l_{x}(t)
                        + \eta^l_{2z}\p_y{\sigma}(t).
\end{eqnarray*}
The $\gamma$ functions are given by
\begin{eqnarray*}
\gamma^l_x(t+3/2\Delta t)  & = &\alpha^l_{1x} \gamma^l_x(t+\Delta t/2) 
                        + \alpha^l_{2x}\dot{e}_{xx}(t+\Delta t/2), \\
\gamma^l_z(t+3/2\Delta t)  & = &\alpha^l_{1z} \gamma^l_z(t+\Delta t/2) 
                        + \alpha^l_{2z}\dot{e}_{zz}(t+\Delta t/2). 
\end{eqnarray*}

%-----------------------------------------------------------------------
\subsection*{Standard linear solid}
%-----------------------------------------------------------------------
The coefficents are
\begin{eqnarray*}
\alpha^l_{1x}  & = & \exp\left(-\frac{d_x(x)\Delta t}
                          {\tau^{\lambda}_{\sigma l}}\right),        \\
\alpha^l_{2x}  & = & \frac{d_x(x)\Delta t}
                  {(\tau^{\lambda}_{\sigma l}
                   \sum_{l=1}^N \frac{\tau^{\lambda}_{\epsilon l}}
                                     {\tau^{\lambda}_{\sigma l}  })
                  }                                                  \\
\alpha^l_{1z}  & = & \exp\left(-\frac{d_z(z)\Delta t}
                          {\tau^{\lambda}_{\sigma l}}\right),        \\
\alpha^l_{2z}  & = & \frac{d_z(z)\Delta t}
                  {(\tau^{\lambda}_{\sigma l}
                   \sum_{l=1}^N \frac{\tau^{\lambda}_{\epsilon l}}
                                     {\tau^{\lambda}_{\sigma l}  })
                  }                                                  \\
\eta^l_{1x}  & = & \exp\left(-\frac{d_x(x)\Delta t}
                          {\tau^{\rho}_{\sigma l}}\right),           \\
\eta^l_{2x}  & = & \frac{d_x(x)\Delta t}
                  {(\tau^{\rho}_{\sigma l}
                   \sum_{l=1}^N \frac{\tau^{\rho}_{\epsilon l}}
                                     {\tau^{\rho}_{\sigma l}  })
                  },                                                 \\
\eta^l_{1z}  & = & \exp\left(-\frac{d_z(z)\Delta t}
                          {\tau^{\rho}_{\sigma l}}\right),           \\
\eta^l_{2z}  & = & \frac{d_z(z)\Delta t}
                  {(\tau^{\rho}_{\sigma l}
                   \sum_{l=1}^N \frac{\tau^{\rho}_{\epsilon l}}
                                     {\tau^{\rho}_{\sigma l}  })
                  }.
\end{eqnarray*}
The profile functions $d_x$ and $d_z$ are
\begin{eqnarray*}
  d_x(x) = (x/L)^2,
  d_z(y) = (z/L)^2,
\end{eqnarray*}
where $L$ is the length of the absorbing layer and 
and we also have
\begin{eqnarray}
\Delta \lambda_l = \lambda_u\left(1-\frac{\tau^{\lambda}_{\epsilon l}}{\tau^{\lambda}_{\sigma l}}\right)
\Delta \rho^{-1} = \rho^{-1}_u\left(1-\frac{\tau^{\rho}_{\epsilon l}}{\tau^{\rho}_{\sigma l}}\right)
\end{eqnarray}
%-----------------------------------------------------------------------
\subsection*{Maxwell solid}
%-----------------------------------------------------------------------
The coefficents are
\begin{eqnarray}
  \alpha_{1x} & = & -\frac{1}{\tau^{\lambda}_0}\exp\left(-\frac{d_x(x)\Delta t}
                          {\tau^{\lambda}_0}\right),              \\
  \alpha_{2x} & = & \frac{d_x(x)\Delta t}
                  {\tau^{\lambda}_0},                             \\
  \alpha_{1z} & = & -\frac{1}{\tau^{\lambda}_0}\exp\left(-\frac{d_z(z)\Delta t}
                  {\tau^{\lambda}_0}\right),                      \\
  \alpha_{2z}  & = & \frac{d_z(z)\Delta t}
                  {\tau^{\lambda}_0}.                             \\
  \eta_{1x} & = & -\frac{1}{\tau^{\rho}_0}\exp\left(-\frac{d_x(x)\Delta t}
                          {\tau^{\rho}_0}\right),                 \\
  \eta_{2x} & = & \frac{d_x(x)\Delta t}
                  {\tau^{\rho}_0},                                \\
  \eta_{1z} & = & -\frac{1}{\tau^{\rho}_0}\exp\left(-\frac{d_z(z)\Delta t}
                  {\tau^{\rho}_0}\right),                         \\
  \eta_{2z}  & = & \frac{d_z(z)\Delta t}
                  {\tau^{\rho}_0}.                                
\end{eqnarray}
We also have
\begin{eqnarray}
  \Delta \lambda 
    & = & \lambda_u,                  \\
  \Delta \rho^{-1} 
    & = & \rho^{-1}_u.
\end{eqnarray}
%-----------------------------------------------------------------------
\subsection*{Plane Waves }
%-----------------------------------------------------------------------
The bulk modulus is considered to be visco-acoustic and is in the
frequency domain given as
\begin{eqnarray}
 K(\omega) = K_0 G(\omega),
                                    \label{eq:12}
\end{eqnarray}
where $K_0$ is a constant to be specified.
The density is also taken to be frequency dependent and complex
\begin{eqnarray}
 \rho(\omega) = \rho_0 G^{-1}(\omega),
                                    \label{eq:13}
\end{eqnarray}

 Consider a horizontally layered medium with bulk-modulus and density
 given by equations \ref{eq:12} and \ref{eq:13}. According to Ursin
 (1983), a downgoing wave in a homogeneous layer with unit amplitude is given by
\begin{eqnarray}
 \exp(-ik_z |z|),
                               \label{eq:16}
\end{eqnarray}
where $z$ is the depth of the layer and the vertical wavenumber $k_z$
is given as:
\begin{eqnarray}
 k_z = \frac{\omega}{c}\cos(\theta).
                               \label{eq:17}
\end{eqnarray}
Here $\theta$ is the angle between the z-axis and the propagation
direction of the wave.
The wave velocity $c$ is given
by
\begin{eqnarray}
 c=\sqrt{\frac{K(\omega)}{\rho(\omega)}} = 
 \sqrt{\frac{K_0}{\rho_0}}G(\omega).
                               \label{eq:18}
\end{eqnarray}
 The reflection coefficient
 accross an interface between two layers (Ursin, 1983) is given by
\begin{eqnarray}
  R = \frac{Z_- - Z_+}{Z_ - + Z_+}, 
                               \label{eq:19}
\end{eqnarray}
where the impedances $Z_-$ and $Z_+$ above and below the interface,
respectively, are given by
\begin{eqnarray}
 Z = \frac{\rho\omega}{k_z} = Z_0 ,
\end{eqnarray}
where $Z_0 = \rho_0 c_0\cos(\theta)$.
Consider the special case where the medium above the interface is
purely acoustic, i.e $G=H=1$, and the medium below the interface is
visco-acoustic, and in addition choose $Z_0$ above the interface equal
to $Z_0$ below the interface. Then $R$ is exactly zero for all incidence
angles.
%------------------------------------------
\subsection*{Maxwell solid}
%------------------------------------------
A particular choice is the classical Maxwell visco-acoustic
medium with $G$ given by \citep{Casula1992}
\begin{eqnarray}
 G(\omega) = \left({1-\frac{i}{\omega\tau}}\right)^{-1},
                                    \label{eq:21}
\end{eqnarray}

We then get for the velocity c
\begin{eqnarray}
 c=\left(1-\frac{i}{\omega\tau}\right)^{-1}\sqrt{\frac{K_0}{\rho_0}},
                               \label{eq:23}
\end{eqnarray}
 and the vertical wavenumber $k_z$ becomes
\begin{eqnarray}
 k_z = \frac{\omega}{c_0} - \frac{i}{c_0\tau}.
                               \label{eq:24}
\end{eqnarray}
The wave are then described by
\begin{eqnarray}
 \exp\left(\frac{-i\omega z}{c_0}\right)\exp(\frac{-z}{c_0\tau}),
                               \label{eq:25}
\end{eqnarray}
 which is a dispersion-free wave with frequency independent
 attenuation.
%------------------------------------------
\subsection*{Standard linear solid}
%------------------------------------------
A particular choice is the standard linear solid visco-acoustic
medium with $G$ given by \citep{Casula1992}
\begin{eqnarray}
 G(\omega) = \frac{1+i\omega\tau_{\epsilon}}{1+i\omega\tau_{\sigma}}
\end{eqnarray}

We then get for the velocity c
\begin{eqnarray}
 c=\frac{1+i\omega\tau_{\epsilon}}{1+i\omega\tau_{\sigma}}
\sqrt{\frac{K_0}{\rho_0}},
\end{eqnarray}

 and the vertical wavenumber $k_z$ becomes

\begin{eqnarray}
 k_z = \frac{\omega}{c_0}\frac{1+i\omega\tau_{\epsilon}}{1+i\omega\tau_{\sigma}}
\end{eqnarray}
The wave are then described by
\begin{eqnarray}
 \exp\left[iz \left(\frac{\omega}{c_0}\right)
      \frac{1+i\omega\tau_{\epsilon}}{1+i\omega\tau_{\sigma}}
     \right].
\end{eqnarray}
which is 
\begin{eqnarray}
 \exp\left[iz\left(\frac{\omega}{c_0}\right)
             \frac{1+\omega^2\tau_{\epsilon}\tau_{\sigma}}
                  {1+\omega^2\tau^2_{\sigma}}
             \right]
 \exp\left[\left(\frac{-z}{c_0}\right)
             \frac{\omega^2(\tau_{\epsilon}-\tau_{\sigma})}
                  {1+\omega^2\tau^2_{\sigma}}
             \right]
\end{eqnarray}

%==========================================================================
\section*{Time dependent density}
%==========================================================================
\cite{Williams2001} gives an expression for dynamic density used for
acoustic waves in porous media:
\begin{eqnarray}
  \rho_{eff}(\omega) = 
\rho_f \left( 
              \frac{\alpha(1-\beta)\rho_s + \beta(\alpha-1)\rho_f 
                    + \frac{i \beta\rho F \eta}{\rho_f\omega\kappa}
              }
              {\beta(1-\beta)\rho_s + (\alpha -2\beta+\beta^2)\rho_f
                    + \frac{i \beta F \eta}{\omega\kappa}
              }
       \right)
\end{eqnarray}
The parameters are
\begin{center}
\begin{tabular}{ c c }
 $\beta$    & porosity                       \\ 
 $\rho_s$   & Mass density of grains         \\  
 $\eta$     & Viscosity                      \\
 $\kappa$   & Permeability                   \\
 $\alpha$   & Tortuosity                     \\
 $F$        & Frequency correction function
\end{tabular}
\end{center}

Multiply by $i\omega$ to get:
\begin{eqnarray}
  \rho_{eff}(\omega) = 
\rho_f \left( 
              \frac{[\alpha(1-\beta)\rho_s + \beta(\alpha-1)\rho_f]i\omega 
                    - \frac{\beta\rho F \eta}{\rho_f\kappa}
              }
              {[\beta(1-\beta)\rho_s + (\alpha -2\beta+\beta^2)\rho_f]i\omega
                    - \frac{ \beta F \eta}{\kappa}
              }
       \right)
\end{eqnarray}
\begin{eqnarray}
  \rho_{eff}(\omega) = 
 \frac{\beta\rho F \eta/\rho_f \kappa}{\beta F \eta/ \kappa}
\rho_f \left( 
              \frac{\frac{[\alpha(1-\beta)\rho_s + \beta(\alpha-1)\rho_f]}{\beta\rho F \eta/ \rho_f \kappa}i\omega 
              -1
              }
              {\frac{[\beta(1-\beta)\rho_s + (\alpha -2\beta+\beta^2)\rho_f]}{\beta F \eta/ \kappa} i\omega
              -1}
       \right)
\end{eqnarray}
\begin{eqnarray}
  \rho_{eff}(\omega) = 
\rho \left( 
              \frac{1-\frac{[\alpha(1-\beta)\rho_s + \beta(\alpha-1)\rho_f]}{\beta\rho F \eta/ \rho_f \kappa}i\omega 
              }
              {1-\frac{[\beta(1-\beta)\rho_s + (\alpha -2\beta+\beta^2)\rho_f]}{\beta F \eta/ \kappa} i\omega
              }
       \right)
\end{eqnarray}

\begin{eqnarray}
  \rho^{-1}_{eff}(\omega) = 
\rho^{-1} {\left( 
              \frac{1-\frac{[\alpha(1-\beta)\rho_s + \beta(\alpha-1)\rho_f]}{\beta\rho F \eta/ \rho_f \kappa}i\omega 
              }
              {1-\frac{[\beta(1-\beta)\rho_s + (\alpha -2\beta+\beta^2)\rho_f]}{\beta F \eta/ \kappa} i\omega
              }
       \right)}^{-1}
\end{eqnarray}

\begin{eqnarray}
  \rho^{-1}_{eff}(\omega) = 
\rho^{-1} {\left( 
              \frac{1-\tau_{\epsilon}i\omega} 
                   {1-\tau_{\sigma}\i\omega}
       \right)}^{-1}
\end{eqnarray}
where
\begin{eqnarray}
  \tau_{\sigma}   & = &\frac{[\alpha(1-\beta)\rho_s + \beta(\alpha-1)\rho_f]}
                            {\beta\rho F \eta/ \rho_f \kappa}, \nonumber\\
 \tau_{\epsilon} & = &\frac{[\beta(1-\beta)\rho_s + (\alpha -2\beta+\beta^2)\rho_f]}
                           {\beta F \eta/ \kappa}. 
\end{eqnarray}
%==============================================================================
\bibliographystyle{seg}  % style file is seg.bst
\bibliography{references}
%==============================================================================
%===============================================================================
\Appendix{A}{The viscoelastic standard linear solid} 
%===============================================================================
Bolzman's generalization of Hook's law to the visco-elastic case is
\citep{Hudson1985}:
\begin{eqnarray}
\sigma_{ij} = \psi_{ijkl}*\dot{e}_{kl},
                  \label{eq:boltzman}
\end{eqnarray}where $\psi_{ijkl}$ is known as the relaxation tensor.
The $*$ denotes convolution defined by
\begin{eqnarray}
a(t)*b(t)=\int_0^t a(t-\tau)b(\tau).
\end{eqnarray}
Integrating \eqref{eq:boltzman} by parts
\begin{eqnarray}
\sigma_{ij}(t)=|_0^t \psi_{ijkl}(t-\tau)e_{kl}(\tau) +\int_0^t \dot{\psi}_{ijkl}(t-\tau)e_{kl}(\tau),
\end{eqnarray}
and using $e(t=0)=0$ I get
\begin{eqnarray}
\sigma_{ij}(t)=\psi(0)_{ijkl}e_{kl}(t) +\int_{0+}^t \dot{\psi}_{ijkl}(t-\tau)e_{kl}(\tau)
\end{eqnarray}
For the Zener model the components of the $\psi_{ijkl}$ tensor have the form
\begin{eqnarray}
  \psi(t)=K\left[1-\frac{1}{N}\sum_{l=1}^N 
            (1-\frac{\tau_{\epsilon l}}{\tau_{\sigma l}})\exp(-t/\tau_{\sigma l})\right]H(t).
\end{eqnarray}
%
where $K_r$ is a relaxed modulus, $N$ is the number of Zener mechanisms,$\tau_{\sigma l}$
and $\tau_{\epsilon l}$ are relaxation times. $H(t)$ is the Heavy side function.
The time derivative of $\psi$ is equal to:
\begin{eqnarray}
\dot{\psi} = \phi(t),
\end{eqnarray}
where $\phi$ is equal to:
\begin{eqnarray}
\phi(t)=
  \frac{1}{N}\sum_{l=1}^N\left[\left(\frac{K_r}{\tau_{\sigma l}}\right) 
            \left(1-\frac{\tau_{\epsilon l}}{\tau_{\sigma l}}\right)\exp(-t/\tau_{\sigma l})\right]H(t).
                \label{eq:phi}
\end{eqnarray}

Using the above we have for the stress:
\begin{eqnarray}
\sigma_{ij}(t)=c_{ijkl}e_{kl} +\int_{0+}^t \phi_{ijkl}(t-\tau)e_{kl}(\tau)
\end{eqnarray}
This is most conveniently written as
\begin{eqnarray}
\sigma_{ij}(t)=c_{ijkl}(t)*e_{kl}(t),
\end{eqnarray}
where
\begin{eqnarray}
 c_{ijkl}(t)=\psi(0)_{ijkl}\delta(t) +\phi_{ijkl}(t),
\end{eqnarray}
%
By definition $\psi(t=0)$ corresponds to the unrelaxed modulus so that we have
\begin{eqnarray}
K_u = \frac{1}{N}\sum_{l=1}^N \frac{\tau_{\epsilon l}}{\tau_{\sigma l}} K_r
\end{eqnarray}
or
\begin{eqnarray}
K_r = \frac{K_u}{\frac{1}{N}\sum_{l=1}^N \frac{\tau_{\epsilon l}}{\tau_{\sigma l}}}
\end{eqnarray}
The $\phi$ function can then be expressed in terms of the unrelaxed moduli:
\begin{eqnarray}
\phi(t) =   
  \sum_{l=1}^N\left[\left(\frac{\exp(-t/\tau_{\sigma l})}{\tau_{\sigma l}\sum_{l=1}^N \frac{\tau_{\epsilon l}}{\tau_{\sigma l}}}\right)
            K_u\left(1-\frac{\tau_{\epsilon l}}{\tau_{\sigma l}}\right)\right]
\end{eqnarray}
Finally, we express $\phi$ as:
\begin{eqnarray}
\phi(t) = \sum_{l=1}^N \phi^l(t)
\end{eqnarray}
where
\begin{eqnarray}
\phi^l(t) = 
  \left(\frac{\exp(-t/\tau_{\sigma l})}{\tau_{\sigma l}\sum_{l=1}^N \frac{\tau_{\epsilon l}}{\tau_{\sigma l}}}\right)
            \Delta K_l
\end{eqnarray}
and $\Delta K_l$ is
\begin{eqnarray}
\Delta K_l = K_u\left(1-\frac{\tau_{\epsilon l}}{\tau_{\sigma l}}\right)
\end{eqnarray}

It is most practical to write the time-dependent visco-elastic constants as
\begin{eqnarray}
  \lambda(t) & = & \lambda_u \delta(t) + \phi_{\lambda}(t), \\
  \mu(t)    &  =  &\mu_u\delta(t)+\phi_{\mu}(t),
\end{eqnarray}
where
%
$\phi_{\lambda}$ is given as:
\begin{eqnarray}
  \phi_{\lambda}(t) = \sum_{l=1}^N 
  \left(\frac{\exp(-t/\tau^{\lambda}_{\sigma l})}{\tau^{\lambda}_{\sigma l}\sum_{l=1}^N \frac{\tau^{\lambda}_{\epsilon l}}{\tau^{\lambda}_{\sigma l}}}\right)
            \Delta \lambda_l
\end{eqnarray}
%
and $\phi_{\mu}$ is given as:
\begin{eqnarray}
  \phi_{\mu}(t) = \sum_{l=1}^N 
  \left(\frac{\exp(-t/\tau^{\mu}_{\sigma l})}{\tau^{\mu}_{\sigma l}\sum_{l=1}^N \frac{\tau^{\mu}_{\epsilon l}}{\tau^{\mu}_{\sigma l}}}\right)
            \Delta \mu_l
\end{eqnarray}
%-----------------------------------------------------------------------------
\subsection*{Q-factors}
%-----------------------------------------------------------------------------
See \cite{Casula1992} for further relations between Q and relaxation times.
The $Q$ values are related to the Fourier transform of the $\lambda$ modulus as
\begin{eqnarray}
  Q^{-1}_{\lambda}(\omega) = \frac{Im \lambda(\omega)}{Re \lambda(\omega)}
\end{eqnarray}
Assuming $\lambda$ is given as
\begin{eqnarray}
\lambda(t) = \lambda_u\delta(t) + \phi_{\lambda}(t).
\end{eqnarray}
The fourier transform of $\lambda$ is given by

\begin{eqnarray}
\lambda(\omega) = \lambda_u + \int_{-\infty}^{\infty} \phi_{\lambda}(t) \exp(-i\omega t).
\end{eqnarray}
The Fourier transform of $\phi_\lambda$ is:

\begin{eqnarray}
\phi_{\lambda}(\omega)=  \frac{1}{N}\sum_{l=1}^N\left(\frac{\lambda_r}{\tau^{\lambda}_{\sigma l}}\right) 
            \left(1-\frac{\tau^{\lambda}_{\epsilon l}}{\tau^{\lambda}_{\sigma l}}\right)
           \int^{+\infty}_0 dt\exp(-i\omega)\exp(-t/\tau^{\lambda}_{\sigma l}).
\end{eqnarray}

The results is:
\begin{eqnarray}
\phi_{\lambda}(\omega)=  \frac{1}{N}\sum_{l=1}^N\left(\frac{\lambda_r}{\tau^{\lambda}_{\sigma l}}\right) 
            \left(1-\frac{\tau^{\lambda}_{\epsilon l}}{\tau^{\lambda}_{\sigma l}} \right)\frac{1}{1+i\omega\tau^{\lambda}_{\sigma l}}.
\end{eqnarray}

The fourier transform of $\lambda$ is then
\begin{eqnarray}
\lambda(\omega) = \lambda_u 
+\frac{1}{N}\sum_{l=1}^N\left(\frac{\lambda_r}{\tau^{\lambda}_{\sigma l}}\right) 
            \left(1-\frac{\tau^{\lambda}_{\epsilon l}}{\tau^{\lambda}_{\sigma l}} \right)\frac{1}{1+i\omega\tau^{\lambda}_{\sigma l}}.
\end{eqnarray}

After some (tedious) algebra one obtains
\begin{eqnarray}
\lambda(\omega) = \lambda_r \frac{1}{N}\sum_{l=1}^N\frac{1+i\omega\tau^{\lambda}_{\epsilon l}}{1+i\omega\tau^{\lambda}_{\sigma l}} 
\end{eqnarray}

Separating into real and imaginary parts, I get

\begin{eqnarray}
\lambda(\omega) & = & \lambda_r 
\frac{1}{N}\sum_{l=1}^N
\frac{1+\omega^2\tau^{\lambda}_{\sigma l}\tau^{\lambda}_{\epsilon l}}
     {1+(\omega\tau^{\lambda}_{\sigma l})^2}  \\\nonumber
                & + & i\lambda_r \frac{1}{N}\sum_{l=1}^N\frac{\omega\tau^{\lambda}_{\sigma l}(\tau^{\lambda}_{\epsilon}/\tau^{\lambda}_{\sigma l}-1)}
                                                              {1+(\omega\tau^{\lambda}_{\sigma l})^2}  \nonumber\\
\end{eqnarray}

We then have
\begin{eqnarray}
Q^{-1}_{\lambda} = \frac{\sum_{l=1}^N {\omega\tau^{\lambda}_{\sigma l}\left(\tau^{\lambda}_{\epsilon l}/\tau^{\lambda}_{\sigma l} -1\right)}
                                           /\left[1 + (\omega\tau^{\lambda}_{\sigma l})^2\right] 
                        }
                        {\sum_{l=1}^N (1 + \omega^2\tau^{\lambda}_{\sigma l}\tau^{\lambda}_{\epsilon l})
                                           /\left[1 + (\omega\tau^{\lambda}_{\sigma l})^2\right]       
                        }
\end{eqnarray}

The results for the frequency dependence of $\mu$ is obtained in exactly the same manner as above:
\begin{eqnarray}
\mu(\omega) = \mu_r \frac{1}{N}\sum_{l=1}^N\frac{1+i\omega\tau^{\mu}_{\epsilon l}}{1+i\omega\tau^{\mu}_{\sigma l}} 
\end{eqnarray}
 and the Q-factor for $\mu$ is
%
\begin{eqnarray}
Q^{-1}_{\mu} = \frac{\sum_{l=1}^N {\omega\tau^{\lambda}_{\sigma l}\left(\tau^{\mu}_{\epsilon l}/\tau^{\lambda}_{\sigma l} -1\right)}
                                           /\left[1 + (\omega\tau^{\mu}_{\sigma l})^2\right] 
                        }
                        {\sum_{l=1}^N (1 + \omega^2\tau^{\mu}_{\sigma l}\tau^{\mu}_{\epsilon l})
                                           /\left[1 + (\omega\tau^{\mu}_{\sigma l})^2\right]       
                        }
\end{eqnarray}
%
In practice we need to relate $Q_l$ and $Q_{\mu}$ to $Q_{\kappa}$. We use the relation
%
\begin{eqnarray}
\kappa(\omega) = \lambda(\omega)+ \frac{2}{3}\mu(\omega).
\end{eqnarray}
%
Splitting into real and imaginary parts
%
\begin{eqnarray}
\kappa_r(\omega) = \lambda_r(\omega)+ \frac{2}{3}\mu_r(\omega).
\kappa_i(\omega) = \lambda_i(\omega)+ \frac{2}{3}\mu_i(\omega).
\end{eqnarray}
%
\begin{eqnarray}
\frac{1}{Q_{\kappa}} = \frac{\kappa_i}{\kappa_r} = \frac{\lambda_i + (2/3)\mu_i}{\lambda_r + (2/3)\mu_r}
\end{eqnarray}
%
%
\begin{eqnarray}
\frac{1}{Q_{\kappa}} = \frac{\kappa_i}{\kappa_r} = \frac{\left(\frac{\lambda_i}{\lambda_r}\right)\lambda_r
                                                   + (2/3)\left(\frac{\mu_i}{\mu_r}\right)\mu_r}
                                                  {\lambda_r + (2/3)\mu_r}
\end{eqnarray}
%
%
\begin{eqnarray}
Q^{-1}_{\kappa} = \frac{Q^{-1}_\lambda \lambda_r + (2/3)Q^{-1}_{\mu}}
                                                  {\lambda_r + (2/3)\mu_r}
\end{eqnarray}
%
We can use the P-wave and S-wave velocities
%
\begin{eqnarray}
\lambda = \rho V^2_p - (2/3)\rho V^2_s \nonumber\\
\mu = \rho V^2_s
\end{eqnarray}
%
\begin{eqnarray}
Q^{-1}_{\kappa} = Q^{-1}_\lambda\left[1-\left(\frac{2}{3}\right)\left(\frac{V_s}{V_p}\right)^2\right]
                  + Q^{-1}_{\mu}\left(\frac{2}{3}\right)\left(\frac{V_s}{V_p}\right)^2
\end{eqnarray}
%
%------------------------------------------------------------------------
\subsection*{Q-model parametrization}
%-------------------------------------------------------------------------------
Q-models can be described by the two relaxation times, $\tau_{\sigma}$
and $\tau_{\epsilon}$. However it is simpler to use the two parameters
$\tau_0$ and $Q_0$ to describe a model.
According to \cite{Casula1992}, Appendix B, we have
\begin{eqnarray*}
  Q(\omega) = Q_0\frac{1+\omega^2\tau_0^2}{2\omega\tau_0}
\end{eqnarray*}
where
\begin{eqnarray*}
  Q_0 = \frac{2\tau_0}{\tau_{\epsilon} -\tau_{\sigma}}, \\
  \tau_0^2 = \tau_{\epsilon}\tau_{\sigma}.
\end{eqnarray*}
$\omega=1/\tau_0$ is the minimum value for $Q(\omega)$, i.e. the
absorption top.
We can now find $\tau_{\sigma}$ and $\tau_{\epsilon}$ in terms of
$\tau_0$ and $Q_0$ as:
\begin{eqnarray*}
  \tau_{\epsilon} = \frac{\tau_0}{Q_0}\left[\sqrt{Q^2_0+1} +1\right], \\
  \tau_{\sigma} = \frac{\tau_0}{Q_0}\left[\sqrt{Q^2_0+1} -1\right].
\end{eqnarray*}
%-------------------------------------------------------------------------------
\subsection*{Effective density} 
%-------------------------------------------------------------------------------
We now assume that the effective density has the following form
\begin{eqnarray}
 \rho^{-1}_{eff}(t) = s(t).
\end{eqnarray}
We also assume that 
\begin{eqnarray}
 s(t) = s(0)\delta(t)+\chi(t)
\end{eqnarray}
So that the inverse of the effective density reads
\begin{eqnarray}
 \rho^{-1}_{eff}(t) = s_u\delta(t)+\chi(t).
\end{eqnarray}
where $s_u = s(0)=\rho^{-1}_u$.

%===============================================================================
\Appendix{C}{The Maxwell visco-elastic solid}
%===============================================================================
According to \cite{Casula1992} the Maxwell  visco-elastic solid
has a modulus given by
\begin{eqnarray}
\lambda(t) = \lambda_u\exp(-t/\tau_0)H(t)
\end{eqnarray}
In the frequency domain one gets
\begin{eqnarray}
\lambda(\omega) = \frac{\tau_0 \omega}{\omega\tau_0 -i}. 
\end{eqnarray}
or,
\begin{eqnarray}
\lambda(\omega) = \frac{\tau_0 i\omega}{i\omega\tau_0 +1}. 
\end{eqnarray}
The function $\phi$ is now:
\begin{eqnarray}
\phi(t) = -\Delta\lambda\frac{1}{\tau_0}\exp(-t/\tau_0),
\end{eqnarray}
where $\Delta\lambda = \lambda_u.$
The Q-value are related to $\tau_0$ by:
\begin{eqnarray}
  \tau_0 = Q(\omega)/\omega. 
\end{eqnarray}
A plane wave 



%===============================================================================
\Appendix{D}{Recursive computation of $\gamma$ and $s$.}
%===============================================================================
\subsection*{Standard Linear Solid}
The $\gamma$'s involve a convolution, which is difficult to compute directly.
However, a recursive relation can be found by considering
\begin{eqnarray*}
\gamma^{\lambda}_x(t+\Delta t) & = &
  \int^{t+\Delta t}_0 d\tau\,
  \frac{1}{\tau^{\lambda}_{\sigma l}
   {\sum_{l=1}^N \frac{\tau^{\lambda}_{\epsilon l}}
   {\tau^{\lambda}_{\sigma l}}} }
  \exp(-\frac{t+\Delta t -\tau}{\tau^{\lambda}_{\sigma}})
  \dot{e}_{xx}(\tau).
\end{eqnarray*}
\begin{eqnarray*}
\gamma^{\lambda}_x(t+\Delta t) & = &
  \frac{1}{\tau^{\lambda}_{\sigma l}{\sum_{l=1}^N \frac{\tau^{\lambda}_{\epsilon l}}{\tau^{\lambda}_{\sigma l}}} }
  \exp(-\frac{\Delta t}{\tau^{\lambda}_{\sigma}})
  \int^{t+\Delta t}_{0} d\tau\,
  \exp(-\frac{t-\tau}{\tau^{\lambda}_{\sigma}})
  \dot{e}_{xx}(\tau).
\end{eqnarray*}
\begin{eqnarray*}
\gamma^{\lambda}_x(t+\Delta t) 
 & = & 
  \frac{1}{\tau^{\lambda}_{\sigma l}\sum_{l=1}^N \frac{\tau^{\rho}_{\epsilon l}}{\tau^{\rho}_{\sigma l}}}
  \exp(-\frac{\Delta t}{\tau^{\lambda}_{\sigma}})
  \int^{t}_0 d\tau\,
  \exp(-\frac{t-\tau}{\tau^{\lambda}_{\sigma}})
  \dot{e}_{xx}(\tau)\\
 & + &
  \frac{1}{\tau^{\lambda}_{\sigma l}\sum_{l=1}^N \frac{\tau^{\rho}_{\epsilon l}}{\tau^{\rho}_{\sigma l}}}
  %\frac{1}{\sum_{l=1}^N \frac{\tau^{\rho}_{\epsilon l}}{\tau^{\rho}_{\sigma l}}}
  \exp(-\frac{\Delta t}{\tau^{\lambda}_{\sigma}})
  \int^{t+\Delta t}_t d\tau\,
  \exp(-\frac{t-\tau}{\tau^{\lambda}_{\sigma}})
  \dot{e}_{xx}(\tau).
\end{eqnarray*}
The second intergral is approximated by
 assuming that $\dot{e}_{xx}(t)$ is constant in the interval $t$ to $t+\Delta t$
\begin{eqnarray*}
\gamma^{\lambda}_x(t+\Delta t) 
 & = & %\frac{1}{\tau_{\epsilon}} 
  \frac{1}{\tau^{\lambda}_{\sigma l}\sum_{l=1}^N \frac{\tau^{\rho}_{\epsilon l}}{\tau^{\rho}_{\sigma l}}}
\exp(-\frac{\Delta t}{\tau^{\lambda}_{\sigma}})
       \int^{t}_0 d\tau\,
       \exp(-\frac{t-\tau}{\tau^{\lambda}_{\sigma}})
       \dot{e}_{xx}(\tau)\\
 & + &
       \frac{1}{\tau_{\epsilon}} \exp(-\frac{\Delta t}{\tau^{\lambda}_{\sigma}})
       \dot{e}_{xx}(t)\int^{t+\Delta t}_t d\tau\,
       \exp(-\frac{t-\tau}{\tau^{\lambda}_{\sigma}})
\end{eqnarray*}
Performing the integral we then get
\begin{eqnarray*}
\gamma^{\lambda}_x(t+\Delta t) 
   & = & \gamma^{\lambda}_x (t)\exp(-\frac{\Delta t}{\tau^{\lambda}_{\sigma}}) \\
   & + & \frac{\tau_{\sigma}}{\tau_{\epsilon}}
         \left[1-\exp(-\frac{\Delta t}{\tau^{\lambda}_{\sigma}})\right]
         \dot{e}_{xx}(t).
\end{eqnarray*}
%
For small $\Delta t \ll 1$ the last equation is also
\begin{eqnarray*}
\gamma^{\lambda}_x(t+\Delta t) 
   & = & \gamma^{\lambda}_x (t)\exp(-\frac{\Delta t}{\tau^{\lambda}_{\sigma}}) \\
   & + & \frac{\Delta t}{\tau_{\epsilon}}\dot{e}_{xx}(t).
\end{eqnarray*}
%------------------------------------------
\subsection*{Maxwell solid}
%------------------------------------------
The $\gamma$'s involve a convolution, which is difficult to compute directly.
However, a recursive relation can be found by considering
\begin{eqnarray*}
\gamma^{\lambda}_x(t+\Delta t) & = &
  -\int^{t+\Delta t}_0 d\tau\,
  \frac{1}{\tau^{\lambda}_0}
  \exp(-\frac{t+\Delta t -\tau}{\tau^{\lambda}_0})
  \dot{e}_{xx}(\tau).
\end{eqnarray*}
\begin{eqnarray*}
\gamma^{\lambda}_x(t+\Delta t) & = &
  -\frac{1}{\tau^{\lambda}_{0}}
  \exp(-\frac{\Delta t}{\tau^{\lambda}_{0}})
  \int^{t+\Delta t}_{0} d\tau\,
  \exp(-\frac{t-\tau}{\tau^{\lambda}_{0}})
  \dot{e}_{xx}(\tau).
\end{eqnarray*}
\begin{eqnarray*}
\gamma^{\lambda}_x(t+\Delta t) 
 & = & 
  -\frac{1}{\tau^{\lambda}_0}
  \exp(-\frac{\Delta t}{\tau^{\lambda}_{0}})
  \int^{t}_0 d\tau\,
  \exp(-\frac{t-\tau}{\tau^{\lambda}_{0}})
  \dot{e}_{xx}(\tau)\\
 & - &
  \frac{1}{\tau^{\lambda}_{0}}
  \exp(-\frac{\Delta t}{\tau^{\lambda}_{0}})
  \int^{t+\Delta t}_t d\tau\,
  \exp(-\frac{t-\tau}{\tau^{\lambda}_{0}})
  \dot{e}_{xx}(\tau).
\end{eqnarray*}
The second intergral is approximated by
 assuming that $\dot{e}_{xx}(t)$ is constant in the interval $t$ to $t+\Delta t$
\begin{eqnarray*}
\gamma^{\lambda}_x(t+\Delta t) 
 & = & 
  -\frac{1}{\tau^{\lambda}_{0}}
\exp(-\frac{\Delta t}{\tau^{\lambda}_{0}})
       \int^{t}_0 d\tau\,
       \exp(-\frac{t-\tau}{\tau^{\lambda}_{0}})
       \dot{e}_{xx}(\tau)\\
 & - &
       \frac{1}{\tau^{\lambda}_{0}} \exp(-\frac{\Delta t}{\tau^{\lambda}_{0}})
       \dot{e}_{xx}(t)\int^{t+\Delta t}_t d\tau\,
       \exp(-\frac{t-\tau}{\tau^{\lambda}_{0}})
\end{eqnarray*}
Performing the integral we then get
\begin{eqnarray*}
\gamma^{\lambda}_x(t+\Delta t) 
   & = & \gamma^{\lambda}_x (t)\exp(-\frac{\Delta t}{\tau^{\lambda}_{\sigma}}) \\
   & + & 
         \left[1-\exp(-\frac{\Delta t}{\tau^{\lambda}_{0}})\right]
         \dot{e}_{xx}(t).
\end{eqnarray*}
%
For small $\Delta t \ll 1$ the last equation is also
\begin{eqnarray*}
\gamma^{\lambda}_x(t+\Delta t) 
   & = & \gamma^{\lambda}_x (t)\exp(-\frac{\Delta t}{\tau^{\lambda}_{\sigma}}) \\
   & + & \frac{\Delta t}{\tau_{0}}\dot{e}_{xx}(t).
\end{eqnarray*}
%==============================================================================
\Appendix{E}{Comparison with CPML}
%==============================================================================
\cite{Komatitsch2007} show that the implementation of the CPML method can
be performed by replacing each spatial derivative with
(Their's equation (16) and (18)):
\begin{eqnarray}
  s_x = \delta(t) - d_x H(t) \exp \left[-(d_x+\alpha_x)t\right].
                                          \label{eq:A-Komatitsch} 
\end{eqnarray}
We compare this with our equation
\begin{eqnarray}
  \lambda(t)/\lambda_u = \delta(t) + \frac{1}{\tau_{\epsilon}}
                         \exp(-t/\tau_{\sigma}) H(t)
                         \left(1-\frac{\tau_{\epsilon}}{\tau_{\sigma}}\right)
                                           \label{eq:A-lambda}
\end{eqnarray}
Comparing equation \eqref{eq:A-Komatitsch} with equation
\eqref{eq:A-lambda} one gets:
\begin{eqnarray}
-d_x         & = & \frac{1}{\tau_{\epsilon}}\left(1-\frac{\tau_{\epsilon}}
                                            {\tau_{\sigma}}\right), \\
d_x +\alpha & = & \frac{1}{\tau_{\sigma}}
\end{eqnarray}
Solving for $\tau_{\epsilon}$ and $\tau_{\sigma}$ one gets
\begin{eqnarray}
  \tau_{\sigma}   & = & \frac{1}{(d_x+\alpha)},\\
  \tau_{\epsilon} & = & \frac{1}{\alpha}.
\end{eqnarray}.
Here 
\begin{eqnarray}
 d_x (x) & = & d_0 \left(\frac{x}{L}\right)^2,
\end{eqnarray}
where $d_0$ is a constant and $L$ is the length of
the PML zone and $x$ is the distance from the start (outer border)
of the PML zone.
\end{document}      

